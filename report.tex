%%%%%%%%%%%%%%%%%%%%%%%%%%%%%%%%%
% 6CCS3PRJ Final Year Individual Project Report
% luke.day@kcl.ac.uk
%%%%%%%%%%%%%%%%%%%%%%%%%%%%%%%%%
\documentclass[11pt]{informatics-report}
\usepackage{color}
\usepackage[square,sort,comma,numbers]{natbib}
\usepackage{listings} %References
\usepackage{semantic}

%%%%%%%%%%%%%%%%%%%%%%%%%%%%%%%%%
% Front Matter - project title, name, supervisor name and date
%%%%%%%%%%%%%%%%%%%%%%%%%%%%%%%%%
\title{6CCS3PRJ Final Year\\\vspace{0.2cm}Lispish to JavaScript compilation}
\author{Daniel Marian Zurawski}
\studentID{1015180}
\supervisor{Christian Urban}

\date{16th November 2012}

\abstractFile{FrontMatter/abstract.tex}
\ackFile{FrontMatter/acknowledgements.tex} %Remove line if you do not want acknowledgements

\begin{document}
\createFrontMatter
\onehalfspacing
\tableofcontents
\doublespacing

%%%%%%%%%%%%%%%%%%%%%%%%%%%%%%%%%
% Report Content
%%%%%%%%%%%%%%%%%%%%%%%%%%%%%%%%%
% You can write each chapter directly here or in a separate .tex file and use the include command.

\chapter{Introduction}
"This is one of the most important components of the report. It should begin with a clear statement of what the project is about so that the nature and scope of the project can be understood by a lay reader. It should summarise everything that you set out to achieve, provide a clear summary of the project's background and relevance to other work, and give pointers to the remaining sections of the report, which will contain the bulk of the technical material"  .... \\
Following the invention of high performance JavaScript compilers such as the Google V8 JavaScript Engine, raised the interest in creating programming language interpreters and compilers that target JavaScript. It enables applications written in other languages, very often high-level languages to be run on any modern web browser. \\

Lisp family of programming languages offers a simplistic and elegant, yet powerful syntax based on parenthesised lists and expressions written in the prefix notation (Polish notation), but they very often require a specific environemnt setup in order for them to run. 

The main motivation behind this project is to investigate how compilers and interpreters for functional languages work and how a translation from a source Lisp language to a target JavaScript language can be performed. The implementation language I am planning to use is Clojure. This project offers a good opportunity to deepen understanding of functional programming and JavaScript and how both can be used to solve complex problems in Computer Science. 
\\
I will design a small Lisp based language called Lispish and I will investigate how such language can be translated to an executable JavaScript. The language will implement a subset of Clojure programming language, which is a modern dialect of Lisp running on the JVM. The compiler will also be implemented in Clojure.
The end result of this project is going to be a small Lisp language (Lispish) and a compiler that can convert the given input string (Lispish program) to a pluggable, executable JavaScript code.

\section{Report Structure}

\chapter{Background}
The background should set the project into context by motivating the subject matter and relating it to existing published work. The background will include a critical evaluation of the existing literature in the area in which your project work is based and should lead the reader to understand how your work is motivated by and related to existing work.

This section provides a throghout background research of the domain of functional programming, Lisp and JavaScript that lead me to the rational behind Lispish design decisions.

\section{Functional programming}
Functional programming is a programming paradigm that differs from imperative programming in a way that it focuses solely on evaluating functions, where one input always results in the same output for a given input (referential transparency). To make this abstraction possible, pure functional languages try to avoid using state and mutability, by ensuring that side effects that could introduce state changes are not possible thus guaranteeing referential transparency.

An example of state is preserving results in variables for later access by other parts of the program. A side effect may result from many different operations such as variable assignments,input/output operations and anything that allows two parts of the program to access the same resource at the same time.

Due to the increase of the demand for parallelisation, as more and more processing cores are added to modern CPUs it is therefore essential that the software we write can be parallelised easily and without the risk of errors that could be caused by race conditions or deadlocks - which are all caused by the notion of mutability. 

The notion of pure functions may sound very impractical for a general purpose programming language, therefore functional languages used by practicioneers such as Clojure allow state, but lexicaly scoped to its own function. 
When state is absolutely necessary in order to improve the performance of an application or expose variable to other parts of the program, Clojure allows for so called "atoms", that improve on the classical notion of a variable, as it is still immutable, but instead an atomic swap operation of the content is performed whenever we want to override its content.

The property of immutability is also preserved for data structures, as each time a data structure is modified, a new copy of such structure is retained therefore leaving the old one in tact. This allows for much better parallelisation, as one part of the program may never modify the same data as the other part of the program, which would lead to inconsistent state.

\section{Lisp}
Lisp is amongst one of the worlds oldest family of programming languages, that has developed several dialects since the original Lisp was published in 1958-1960 by John McCarthy. [citation here]
Lisp languages differ from other programming languages in its few original concepts, notably treating code as data, s-expressions, parenthesized Polish prefix notation and lambda expressions.

The exact expansion of the Lisp acronym is List Processing which has its practical reasons - Lisp source code is written as lists, formally called S-Expressions [reference here]

To illustrate how a valid s-expression would look like compared to an equivalent C expression, here is an example:

\begin{lstlisting}
1 == (1 * 1)
\end{lstlisting}
in C is equivalent to

\begin{lstlisting}
(= 1 (* 1 1))
\end{lstlisting}
in Lisp's s-expression based prefix notation.


\subsection{Compiling Lispish using a dialect of Lisp}
The decision to use Clojure to write a compiler for my Lisp language comes from the fact that there are large advantages of using Lisp to compile Lisp.
The nature of Lisp and it's s-expressions allows us to build efficient recursive descent parsers that can take the advantage of the already present functions in our implementation language, Clojure.

Some of the typical complexities that we would encounter when trying to implement a Lisp compiler using a non-lisp imperative language such as C include having to determine if a given expression is an s-expression (list) or a symbol and then breaking the input down into its atomic form of tokens to then build a Syntax Tree (ST) or an Anotated Syntax Tree (AST).
In our case, our input s-expressions can be treated as a ST on its own and thanks to the in-built functions, we can greatly simplify the compiler by making it less error prone.

For example, any s-expression can be essentially evaluated and type-checked using the inbuit "symbol?" or "list?" to determine if the given s-expression yields to a symbol or a list. If an input is a list, that means we have come across another s-expression and each element in the list has to be separetly evaluated. 

Modern dialects of Lisp, such as Clojure, target the Java Virtual Machine making them very portable and pluggable into an existing Java applications.
Other Lisp languages are very often compiled to another target language, such as C or JavaScript that can be then run on a variety of machines. 

\section{JavaScript as a target language for the Lispish language}
The rationale behind selecting JavaScript as the target language is the fact that JavaScript can be executed on almost all of the Internet enabled devices, as long as they have a web browser. Percentage of JS enabled devices as of date: [insert reference here].

Our small dialect of Lisp (Lispish) language will allow generating pluggable JavaScript code. 
From this follows the fact that applications written in Lispish can be executed in environments where the JVM or Clojure is not present, as the generated code will be a standard JavaScript.
In theory our language could even be used as a Domain Specific Language (DSL) for JavaScript applications, as long as the code would be evaluated by our compiler in a Clojure JVM environment.

JavaScript offers a great opportunity as a target language for any high-level programming language primarily due to two reasons - it's portability and performance. 

\subsection{Similarities}
JavaScript is a prototype based, objected-oriented language that due to its great flexibility and full support for lambda expressions can also be classified as a functional language.

\begin{lstlisting}
// Attach event listener to the argument
var assign-event-listener = function(x) {
  x.addEventListener("load", function() { 
  	alert("All done"); 
   }, false)
};
\end{lstlisting}

Above example illustrates a stored function that takes a reference to a web browsers window as an argument and attaches an event listener to it. The listener then takes two arguments, a string descripting the event - here "load" and the callback function - here an anonymous function that displays an alert "All done" that gets displayed after the desired event is triggered.
To now show the similary between how the same expression could look like in Lisp, take for instance:

\begin{lstlisting}
(defn assign-event-listener [window]
  "Attach event listener to the argument"
  (window (addEventListener 
  	   	  "load" 
		  (fn [x] (alert "All done")))))
\end{lstlisting}

Both of the expressions make use of nested functions and thus take the advantage of the lambda calculus. This abstraction can be also one-to-one mapped when peforming the compilation from a Lisp to JavaScript and thus simplyfing the compiler.

\subsection{JavaScript performance}
The invention of the V8 Google JavaScript Engine made JavaScript stand out from other dynamic languages by making it faster than other dynamically types languages such as for e.g. Python [reference required].

Due to the fact that Lispish compiles to JavaScript, the generated code can be treated with various optimisation techniques, including the Google Closure compiler that minimises and optimises the code, by compiling the readable, yet verbose version of the JavaScript code, to a less readable but highly optimised JS code. 

\subsection{Portability}
JavaScript interpreters are present on majority of consumer devices and are present in all of the modern web browsers. It is the basis of Rich Internet Applications and is now not only present on the front end of the web browser, but also servers as a language of choice for back ends.
Most notable examples include Microsoft's cloud platform Windows Azure that operates using JavaScript both on the front end and as well as the back end, making use of the Node.Js framework for producing highly asynchronous web applications. [reference required]

\section{Existing Lisp to JS compilers}
There already exists a number of similar projects, that each tries to solve the problem in a slightly different way, although there exists only one mature compilar that can actually generate an executable JavaScript code and it's called ClojureScript.

\subsection{ClojureScript}
ClojureScript is a Clojure to JavaScript compiler that can already generate code that can be executed in the browser and although there are examples of companies using ClojureScript for their production applications, it is difficult to operate as it requires to execute a chain of operations, including starting a JavaScript program before the Clojure code can be compiled.
ClojureScript also takes the idea further and utilises Google Closure compiler to optimise the code to remove code that can be reduced, thus making it run faster, but this approach also sufers from the fact that the Closure optimising compiler very often breaks the JavaScript code that was compiled from Clojure.

\subsection{LiScript}
FILL IN TEXT HERE

\subsection{clojurejs}
Yet another implementation of the same concept as ClojureScript, although does not support tail recursion and lazy evaluation essentially making it a lot less appealing to the community.

\include{Chapters/Body}
\chapter{Design \& Specification}

As previously described, the project aims to create an extensible Lispish to JavaScript compiler. 
In order to ... we need to formalise our input language Lispish to clearly define the possible constructs that we allow in our program. 
As Lispish defines a subset of an existing language, it is therefore even more important to be clear on what it possible and what is not. 

\section{Designing the Lispish language}

Lispish is a dynamically typed, functional language that implements a call-by-value strategy just as its superset Clojure.

The formal description of Lispish behaviour will be described using transition systems.

\subsection{Grammar of Lispish}

$F \Coloneqq \ (let \ [x \ F] \ (F)) \\
| \	(if \ (F) \ F_1 \ F_2) \\
| \	(defn \ name \ [args*] \ (F)) \\
| \	(fn \ [arg] \ (F)) \\
| \	(cond \ (F_0) \ F_1 \ (F_2) \ F_3) \\
| \     T \\
| \     X
$

where
\\
$X \Coloneqq T $
\\
$T \Coloneqq \ () \ \vert \ N \ \vert \ B \ \vert \ s $
\\
$N \Coloneqq n \ \vert \ (op \ N \ N)  $
\\
$B \Coloneqq b\ \vert \ (bop \ t1 \ t2) \ $
\\
$op \Coloneqq \ + \ \vert \ - \ \vert \ * \ \vert \ /$
\\
$bop \Coloneqq \ > \vert < \vert =$
\\
$s \Coloneqq \ String $
\\
$n \Coloneqq \ Integer $
\\
$b \Coloneqq \ true \ \vert \ false $
\\
$() \Coloneqq \ List $

\subsection{Evaluation relations (Big-Step Semantics)}

\[
\inference[integer]{}
{n \ $$\Downarrow$$ \ n}
\]
\[
\inference[list]{}
{(n) \ $$\Downarrow$$ \ n}
\]
\[
\inference[let]{t_0 \ $$\Downarrow$$ \ v}
{(\text{let } \ {[x \ (t_0)]} \ (t_1)) \ $$\Downarrow$$ \ t_1 {[}x \ $$\mapsto$$ \ v{]} }
\]
\[
\inference[if true]{{t_0} \ $$\Downarrow$$ \ \textsc{True} & t_1 \ $$\Downarrow$$ \ v}
{(if \ (t_0) \ t_1 \ t_2) \ $$\Downarrow$$ \ v) }
\]
\[
\inference[if false]{t_0 \ $$\Downarrow$$ \ \textsc{False} & t2 \ $$\Downarrow$$ \ v}
{(if \ (t_0) \ t_1 \ t_2) \ $$\Downarrow$$ \ v) }
\]
\[
\inference[cond]{t_0 \ $$\Downarrow$$ \ \textsc{False} & t_1 \ $$\Downarrow$$ \ \textsc{True} & t_3 \ $$\Downarrow$$ \ v}
{(cond \ (t_0) \ t_2 \ (t_1) \ t_3) \ $$\Downarrow$$ \ v) }
\]
\[
\inference[defn]{ {t_0 \ [x]} \ $$\Downarrow$$ \ v}
{(defn \ s \ {[x]} \ (t_0)) \ $$\Downarrow$$ \ s \ $$\mapsto$$ \ v }
\]
\[
\inference[fn]{ {t_0 \ [x]} \ $$\Downarrow$$ \ v}
{(fn \ {[x]}  \ (t_0) \ $$\Downarrow$$ \ v }
\]

\section{Development methodology}
In order to streamline the process of development of the compiler, I have decided to use the Test Driven Development (TDD) methodology that emphasizes on building small units of functionalities that can be invidually tested by designeted unit tests. 

Clojure allows developers to create programs using the REPL (Read Evaluation Print Loop), which is characteristic feature in new dynamic programming languages. It allows you to write your functions, evaluate them and get an instant result from an interpreter that interacts with your code. This in essence reduces the amount of unit tests that have to be implemented for trivial functions in a TDD project. 
REPL is a great resource for rapid development and prototyping of functions, but also ensuring that they yield the right result before the project as a whole is compiled.

\subsection{Unit tests}
Even though Clojure provides REPL, it is still important to develop a regression testing suite that ensures whenever the compiler is modified, it can still compile and yield the same result for old programs.
To do this, I will use clojure.test API that provides a set of macros for evaluating forms and ensuring they yield the expected result. 

\section{Compiling Lispish to JavaScript (Compiler design??)}

\subsection{Compilation pipeline (use case/state machines?)}

\begin{figure}[hb]
	\centering
	\includegraphics{Graphics/test.jpg}
	\caption[Abstract \textit{Lispish to JavaScript} compilation.]
   {Abstract figure of \textit{Lispish to JavaScript} compilation.}
\end{figure}

The compiler in its simplest form will perform a one to one translation in-line translation from Lispish to JavaScript. 
The input source will be treated by a macro function that will prevent the code from being evaluated and it will pass it along down the pipeline to its respective emitters as ilustrated on figure [NUMBER HERE]. Code will be treated as data and I will use the prefix notation to my advantage, treating each expression as its respective node in a parse tree. 
\chapter{Implementation and Testing}
Following the formal definition of the Lispish language and briefly describing the operations of the compiler, we shall dive into the construction of it. Along the way, I will explain the concepts behind most of the mechanics of the compiler. 

In the following sections, we will also test the implementation by means of examples of an actual compilation. 
At the end, we will look into automating the tests by means of using a Clojure testing API.  

\section{Building the compiler}
This section will describe the operations of the compiler and the fundamental concepts behind how the compiler translates the input Lispish code to JavaScript. 

Lispish translator is implemented as a single pass compiler.
This design decision comes from the fact that the compiler has been implemeneted in Clojure, which is a strictly functional language and in order to sustain the immutability property throughout the compiler and not violate common idioms, the compiler avoids using state at all costs. It was therefore difficult to perform multi-pass compilation over the same code, as it is done in other commercial compilers. 
As a consequence of this, the entire implementation is built around recursively invoking a set a functions, which at the end fold to yield a JavaScript string as a result. 



\subsection{Recursive Expansion}
The main idea behind the Lispish compiler's implementation is recursive expansion.
The compiler breaks down each s-expression that it comes across into its primitives until there is no more work to be done. It then builds up the result in layers as the recursion folds upwards. 

Figure ~\ref{fig:recursive_expansion_flowchart} illustrates the flow chart of the compiler. It covers most of the operations of the compiler, except for the details on how multi-arity s-expressions are handled.  

\begin{figure}[!htbp]
	\centering
	\includegraphics{Graphics/compilation_flow_chart.png}
	\caption[yadayada]
   {Flow chart of \textit{Lispish to JavaScript} compilation.}
  \label{fig:recursive_expansion_flowchart}
\end{figure}

\subsection{Forms with multiple arity}

In order to solve the multiple arity problem, where for instance a \texttt{(cond )} form can take multiple condition/true-form expression tuples and each one of them has to be compiler to a JavaScript string, map and reduce constructs have been used. 

\subsubsection{Map}
The idea behind the map operation is to apply a function that takes one argument, to all of the elements in a collection and return a new collection with results of each application of the aforementioned function. 
A simple example of Map is 

\begin{verbatim}
(map (fn [x] (+ x 1)) 
	 [0 1 2 3 4 5])
\end{verbatim}
that yields 
\begin{verbatim}
[1 2 3 4 5 6]
\end{verbatim}
as a result

\subsection{Reduce}
Reduce is a function that takes a function, an optional value (or an s-expression) and a collection as an argument. It reduces or in other words folds a given collection (and an optional value) through the application of a function to a collection, to a single result. 
\begin{verbatim}
(reduce 
   str
   1
	 [1 2 3])
\end{verbatim}
that yields 
\begin{verbatim}
"1123"
\end{verbatim}
as a result. The collection of numbers has been reduced to a string, as each number was converted to a string and then a string of the collection has been produced.
If we would to map a \texttt{str} function over the collection of \texttt{[1 2 3]}, it would result in a new collection containing all of the elements of the old collection converted to a string, namely the list \texttt{("1" "2" "3")}.

To now put the map reduce constructs into perspective with Lispish, figure ~\ref{fig:emit-cond-code} illustrates how a multiple arity cond (allowing practically unbound list of tests) is implemented.


\begin{figure}[ht]
\begin{verbatim}
(defn emit-cond [head [name & rest]]
  (let [rev (reverse (partition 2 rest))]
    (reduce
    	(fn [a b] (str "(" (emit (first b)) "?" (emit (second b)) ":" a ")"))
        (str (emit (second (first rev))))
        (drop 1 rev))))
\end{verbatim}
\caption{emit-cond source code}
\label{fig:emit-cond-code}
\end{figure}


Given an arbitrary number of \texttt{(test) result} tuples for the input 
\texttt{(cond )}, the \texttt{(emit-cond)} form first partitions the input into test and expression tuples, then reverses the tuples, so that the originally last one appears at the front, allowing us to perform a right reduce (right fold) and then binds it to a local \texttt{rev} variable. 
For example, if \texttt{(emit-cond )} is invoked with the following arguments:

$$ \texttt{(< 5 2) false (> 3 2) true :else false} $$

the content of the locally scoped \texttt{rev} will be 

$$ \texttt{((:else false) ((> 3 2) true) ((< 5 2) false))} $$

The reduce function then applies the anonymous function to the first value, which is the result of \texttt{(str (emit (second (first rev))))}, which in this example happens to be the \texttt{false} symbol, as it is grabbed from the first tuple \texttt{(:else false)} as the second element. Reduce is then applied to the second, third etc. element of the collection, in this case the \texttt{((> 3 2) true) ((< 5 2) false)}, whilst the overall result is accumulated in \texttt{a}.

\begin{figure}[ht]
\centering
\begin{tabular}{ r | l || l }
1. & \texttt{a: false} & \texttt{b: ((> 3 2) true)} \\
2. & \texttt{a: ((3>2)?true:false)} & \texttt{b: ((< 5 2) false)} \\
3. & \texttt{a: ((5<2)?false:((3>2)?true:false))} & b:
\end{tabular}
\caption{Reduction of a cond with multiple arguments}
\label{fig:emit-cond-expansion}
\end{figure}


Figure ~\ref{fig:emit-cond-expansion} illustrates a table of how each reduction step is performed in terms of the two arguments of the function passed to reduce. Variable \texttt{a} accumulates the overall result, whilst \texttt{b} is the current element of the \texttt{(cond )} that is being converted to JavaScript ternary expression.

\section{Testing}
The section above described the operations that are part of the compilation, but they did not provide any examples of an actual compilation. 
In this section we will take a look at some examples of how our Lispsh to JavaScript compiler works. 
To illustrate the compilation, I will demonstrate the output of the recursive expansion that the compiler performs on the given Lispish program string. 
Each line of the compilation trace will correspond to a level in the recursion. The recursion folding will be done implicitly, therefore it does not appear in the compilation traces. 

Let's begin our tests by a simple arithmetic expression:
\begin{verbatim}
lispish.core> (lisp-to-js "(+ 2 2)")
Emit Lispish:  (+ 2 2)
Emit-list head:  + , tail:  (2 2)
Emit-op, head:  + , tail:  (2 2)
Emit Lispish:  2
Emit Lispish:  2
"(2+2)"
\end{verbatim}
As we can see, our recursion begins with passing the Lispish source code to a lisp-to-js macro, which then begins the recursion by invoking the initial emit step.
At first, our s-expression is of the form (+ 2 2), which is a list. This means that the compiler has to expand the list and emit each individual expression within it. It begins by evaluating the head of the list, which happens to be an "op" operator, in this case the "+" sign. 
It therefore passes the head of the previous s-expression (the "+" sign), as well as the remaining part of the expression (2 2) to emit-op. 
Emit-op outputs the corresponding JavaScript by first mapping the top-level recursion emit function to each element inside of the tail list (2 2) which reaches the bottom of the recursion in one step each and then reduces the result of this to a string concatenated with the operator in the middle.
The same procedure is repeated for all of the "op", as well as "bop" expressions.

In the next sub section, we will have a look at a more complex example of generating a named function that is implemnted largely with Abstract Structural Binding.

\subsection{Abstract Structural Binding}
Abstract Structural Binding in simple words means de-structuring. It allows for de-structuring any data structure to a corresponding argument in function parameters or a let form, creating locally scoped bindings.
For example, if we define a let as follows:
\begin{verbatim}
(let [[x1 x2] [1 2]])
\end{verbatim}
"x1" will yield 1 and "x2" will yield 2.
The same principle is true for a function.
If our function accepts one parameter which is a collection:

\begin{verbatim}
(defn test [[x1 x2]] (println x1 x2))
(test [1])
\end{verbatim}
and it binds the first two elements of the collection to x1 x2, in the above case, x1 will yield 1 and x2 null.

Lispish uses de-structuring for generating all of its forms ($F$):
\begin{verbatim}
lispish.core> (lisp-to-js (defn square [x] (* x x)))
Emit Lispish:  (defn square [x] (* x x))
Emit-list head:  defn , tail:  (square [x] (* x x))
Emit-forms, head:  defn , full expression:  (defn square [x] (* x x))
Emit-defn, name:  square , arg:  x , arg tail:  nil , rest:  ((* x x))
Emit Lispish:  ((* x x))
Emit Lispish:  (* x x)
Emit-list head:  * , tail:  (x x)
Emit-op, head:  * , tail:  (x x)
Emit Lispish:  x
Emit Lispish:  x
"function square(x) {(x*x)}"
\end{verbatim}

In order to split the "defn" expression into into its respective elements, the emit-defn function that gets invoked by emit-list (after determining the head of list to be "defn") performs a structural binding of the function arguments. The bindings are then used to generate the equivalent JavaScript code. 

Lets look at the signature of the emit-defn function:
\begin{verbatim}
(defn emit-defn [type [defn name [arg & more] & rest]]
  )
\end{verbatim}
as we can see, the function takes 4 arguments and 2 optional tail arguments that can be a list of an arbitrary length. The "type" argument is simply a convenience placeholder for the head of the whole expression.
The actual expression begins to bind from the [defn name [arg \& more] \& rest] arguments. 
The optional more in the arguments list of arguments allows for an arbitrary length of the named function arguments and the optional rest is for the expression that follows the named function.

This structure can be then reused to output the corresponding JavaScript as follows:

\begin{verbatim}
(str "function "
  // Name of the function 
  name "("
    (if (nil? more) 
      // If arguments are not a list, then output just single argument
      arg 
      // Else output that list of arguments, separated by a comma
      (str arg ", " (clojure.string/join ", " more))
    )
    // Close the arguments parenthesis and begin function body
    ") {"
    // Emit function body
    (emit rest)
    // Close function body
    "}"
)
\end{verbatim}

\subsection{Deploying and Using Lispish}
The end goal of this project was to be able to compile a source Lispish program to an equivalent JavaScript program.
It is, however, not ideal to have to perform compilation in an interactive REPL, where Clojure environment is set up. 

To solve this problem, the Lispish compiler is compiled as a standalone JAR file that can be executed in any environment equipped with the Java Runtime Environment. This is possible as the JAR file bundles the Clojure language itself, as well as all of its dependencies and our Lispish compiler. It exposes the application through a simple static main method, which serves as an entry point to programs execution, similarly to standard Java applications. 

There are three simple ways to compile a Lispish program to JavaScript. The first method is to execute the Lispish jar file and provide simple source code as a command line argument:

\begin{verbatim}
danielzurawski$ java -jar lispish-1.0.jar "(+ 2 2)"Emit Lispish:  (+ 2 2)
Emit-list head:  + , tail:  (2 2)
Emit-op, head:  + , tail:  (2 2)
Emit Lispish:  2
Emit Lispish:  2
(2+2)
\end{verbatim}

Given as an input a prefix s-expression of (+ 2 2), the program yields an expected result, which is an equivalent in-fix (2+2).

This approach is fine for trivial examples that do not span across multiple lines, it is however not optimal when we want to compile a Lispish program file to an equivalent JS. 
In order to compile a Lispish source file to an equivalent JavaScript source file, our compiler accepts two command line options:

\begin{verbatim}
["-in" "--input" "REQUIRED: Path to Lispish source code."]
["-out" "--output" "OPTIONAL: Path to JavaScript output file."]
\end{verbatim}

\texttt{-in} or equivalently  \texttt{--input}, that should follow with a path to a Lispish source file, as well as an optional 
\texttt{-out} or equivalently \texttt{--output}, that should follow with the name of the output source file. 

To demonstrate how compilation of one source file to another is performed, here is the content of a sample "test.lispish" file:

\begin{verbatim}
danielzurawski$ more test.lispish
(+ 2 2)
\end{verbatim}

We can then execute the compiler passing in the -in and -out arguments, as follows:

\begin{verbatim}
danielzurawski$ java -jar lispish-1.0.jar -in /Users/danielzurawski/git/lispish/test.lispish -out test.js
Emit Lispish:  (+ 2 2)
Emit-list head:  + , tail:  (2 2)
Emit-op, head:  + , tail:  (2 2)
Emit Lispish:  2
Emit Lispish:  2
danielzurawski$
\end{verbatim}

Our "-in" argument is an absolute path to the "test.lispish" file that we printed in the code snippet above and our "-out" argument is the name of the file to be generated, to which the compiler will yield result. 
The compiler will print out all of the computation steps to the console, but the final result that is the JavaScript output will be written to a file.

Now, if we check the content of test.js, we can see
\begin{verbatim}
danielzurawski$ more test.js
(2+2)
danielzurawski$
\end{verbatim}
that the test.js file yields the compiled JavaScript source code. 

This however is a rather trivial example, that does not span accross multiple lines, so lets try something a bit more sophisticated:

The following snippet illustrates the translation of a recursive Ackermann Function [REFERENCE HERE ] from a Lispish source code, to its equivalent JavaScript. 

\begin{verbatim}
danielzurawski$ java -jar lispish-1.0.jar -in /Users/danielzurawski/git/lispish/test.lispish -out test.js
Emit Lispish:  (defn ackermann [m n] (cond (= m 0) (+ n 1) (= n 0) (ackermann (- m 1) 1) :else (ackermann (- m 1) (ackermann m (- n 1)))))
Emit-list head:  defn , tail:  (ackermann [m n] (cond (= m 0) (+ n 1) (= n 0) (ackermann (- m 1) 1) :else (ackermann (- m 1) (ackermann m (- n 1)))))
Emit-forms, head:  defn , full expression:  (defn ackermann [m n] (cond (= m 0) (+ n 1) (= n 0) (ackermann (- m 1) 1) :else (ackermann (- m 1) (ackermann m (- n 1)))))
Emit-defn, name:  ackermann , arg:  m , arg tail:  (n) , rest:  ((cond (= m 0) (+ n 1) (= n 0) (ackermann (- m 1) 1) :else (ackermann (- m 1) (ackermann m (- n 1)))))
Emit Lispish:  ((cond (= m 0) (+ n 1) (= n 0) (ackermann (- m 1) 1) :else (ackermann (- m 1) (ackermann m (- n 1)))))
Emit Lispish:  (cond (= m 0) (+ n 1) (= n 0) (ackermann (- m 1) 1) :else (ackermann (- m 1) (ackermann m (- n 1))))
Emit-list head:  cond , tail:  ((= m 0) (+ n 1) (= n 0) (ackermann (- m 1) 1) :else (ackermann (- m 1) (ackermann m (- n 1))))
Emit-forms, head:  cond , full expression:  (cond (= m 0) (+ n 1) (= n 0) (ackermann (- m 1) 1) :else (ackermann (- m 1) (ackermann m (- n 1))))
Emit Lispish:  (= m 0)
Emit-list head:  = , tail:  (m 0)
Emit-op, head:  = , tail:  (m 0)
Emit Lispish:  m
Emit Lispish:  0
Emit Lispish:  (+ n 1)
Emit-list head:  + , tail:  (n 1)
Emit-op, head:  + , tail:  (n 1)
Emit Lispish:  n
Emit Lispish:  1
Emit Lispish:  (= n 0)
Emit-list head:  = , tail:  (n 0)
Emit-op, head:  = , tail:  (n 0)
Emit Lispish:  n
Emit Lispish:  0
Emit Lispish:  (ackermann (- m 1) 1)
Emit-list head:  ackermann , tail:  ((- m 1) 1)
Emit-forms, head:  ackermann , full expression:  (ackermann (- m 1) 1)
emit-recur, head: ackermann , name:  ackermann , args:  (- m 1) , more:  (1)
Emit Lispish:  (- m 1)
Emit-list head:  - , tail:  (m 1)
Emit-op, head:  - , tail:  (m 1)
Emit Lispish:  m
Emit Lispish:  1
Emit Lispish:  1
Emit Lispish:  (ackermann (- m 1) (ackermann m (- n 1)))
Emit-list head:  ackermann , tail:  ((- m 1) (ackermann m (- n 1)))
Emit-forms, head:  ackermann , full expression:  (ackermann (- m 1) (ackermann m (- n 1)))
emit-recur, head: ackermann , name:  ackermann , args:  (- m 1) , more:  ((ackermann m (- n 1)))
Emit Lispish:  (- m 1)
Emit-list head:  - , tail:  (m 1)
Emit-op, head:  - , tail:  (m 1)
Emit Lispish:  m
Emit Lispish:  1
Emit Lispish:  (ackermann m (- n 1))
Emit-list head:  ackermann , tail:  (m (- n 1))
Emit-forms, head:  ackermann , full expression:  (ackermann m (- n 1))
emit-recur, head: ackermann , name:  ackermann , args:  m , more:  ((- n 1))
Emit Lispish:  m
Emit Lispish:  (- n 1)
Emit-list head:  - , tail:  (n 1)
Emit-op, head:  - , tail:  (n 1)
Emit Lispish:  n
Emit Lispish:  1
\end{verbatim}

The ackermann function is a fairly complicated function that spans across multiple lines and in order to produce an equivalent JavaScript code, a fair amount of recursive calls has to be performed, thus the trace of execution is a rather lengthy one. 

If we now examine the test.js file, it yields:

\begin{verbatim}
danielzurawski$ more test.js
function ackermann(m, n) {if((m==0)) { return (n+1) }else if((n==0)){ return ackermann((m-1), 1) }else if(true){ return ackermann((m-1), ackermann(m, (n-1))) }}
\end{verbatim}

We can now test whether the code can be executed in a JavaScript environment by testing it in, for example, Google Chrome Console: 

\begin{verbatim}
> function ackermann(m, n) {if((m==0)) { return (n+1) }else if((n==0)){ return ackermann((m-1), 1) }else if(true){ return ackermann((m-1), ackermann(m, (n-1))) }}
> ackermann(1, 3)
5
\end{verbatim}

As we can see, the Google Chrome web browsers console can evaluate the function and when executed with parameters, it yields the right result. 

\subsection{Automating tests with clojure.test API}
In order to ensure that the compiler is naturally expanded and all of the of regression tests are performed whenever a new language construct is added, I have decided to use the Test Driven Development methodology to approach this project. 
The tool to support me in the task of TDD I used was the clojure.test API.
clojure.test API [REFERENCE HERE] is a unit testing framework that provides a set of in-built forms, particularly the "is" macro that allows to perform boolean assertions on arbitrary expressions. 

\begin{quote}
\begin{verbatim}
(deftest factorial-example
  (is (= "function factorial(n) {if(n<2) { return 1 } else { return (n*factorial(((n-1)))) }}"
         (lisp-to-js (defn factorial [n] (if (< n 2) 1 (* n (factorial (- n 1)))))))))
\end{verbatim}
\end{quote}


\include{Chapters/ProfessionalIssues}
\chapter{Results/Evaluation}

\section{Section Heading}
\chapter{Conclusions and Future Work}

Functional compilers provide an elegant alternative to compilers written in imperative languages. Nevertheless, it is not trivial to implement one given the relative unpopularity of functional programs. 

The translator in its current state could be a good foundation for initiating a collaboration with the open source community that could have interest in extending it. 

It was not in the scope of this project to provide any formal proofs of the correctness of the translations. This, however, would be a very achievable target, mainly due the functional properties of the compiler -- namely the one-input/one-output property. 
Providing formal proofs for the correctness of the translator and even further, of the generated translations could be a good project for a Master's Thesis. 

\section{Future work}
\subsection{Macros}
It is typical for Lisp languages to provide a way to define new constructs in terms of already existing language constructs.
For this to happen, a language needs to support macros which 
provide a way to extend the language at compile time. 
Due to time constraints, I was unable to perform sufficient research into how to provide the flexibility of macros, while still being able to parse the code correctly. 

\subsection{Parser}
In order for the Lispish translator to be a true compiler, it would need a parser that can decide whether the input string, that is the Lispish program, is in fact valid. As mentioned in the previous chapter, the current compiler does not provide any error detection facilities and this therefore, would be the first step to proper error handling. 

% TODO possible rewrite
%\subsection{JavaScript validator}
%For Lispish to be truly useful, its translator would need to have JavaScript validation in place. The parser used for validation could flag, for example, at which point there is an error in the generated JavaScript code and therefore simplify the task of debugging the translated code. 
%
%JavaScript that is not syntactically correct due to the faulty input Lispish is only going to decrease the productivity of the developer, which goes against the core idea of building an abstraction over an imperative language.


%%%%%%%%%%%%%%%%%%%%%%%%%%%%%%%%%
% References
%%%%%%%%%%%%%%%%%%%%%%%%%%%%%%%%%
\bibliographystyle{plain}
\bibliography{references}
\addcontentsline{toc}{section}{Bibliography}

%%%%%%%%%%%%%%%%%%%%%%%%%%%%%%%%%
% Appendices
%%%%%%%%%%%%%%%%%%%%%%%%%%%%%%%%%
\appendix
\include{Appendices/appendix}
\include{Appendices/UserGuide}
\include{Appendices/SourceCode}
\end{document}
