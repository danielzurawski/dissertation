\chapter{Conclusions and Future Work}

Functional compilers provide an elegant alternative to compilers written in imperative languages. Nevertheless, it is not trivial to implement one given the relative unpopularity of functional programs. 

The translator in its current state could be a good foundation for initiating a collaboration with the open source community that could have interest in extending it. 

It was not in the scope of this project to provide any formal proofs of the correctness of the translations. This, however, would be a very achievable target, mainly due the functional properties of the compiler -- namely the one-input/one-output property. 
Providing formal proofs for the correctness of the translator and even further, of the generated translations could be a good project for a Master's Thesis. 

\section{Future work}
\subsection{Macros}
It is typical for Lisp languages to provide a way to define new constructs in terms of already existing language constructs.
For this to happen, a language needs to support macros which 
provide a way to extend the language at compile time. 
Due to time constraints, I was unable to perform sufficient research into how to provide the flexibility of macros, while still being able to parse the code correctly. 

\subsection{Parser}
In order for the Lispish translator to be a true compiler, it would need a parser that can decide whether the input string, that is the Lispish program, is in fact valid. As mentioned in the previous chapter, the current compiler does not provide any error detection facilities and this therefore, would be the first step to proper error handling. 

% TODO possible rewrite
%\subsection{JavaScript validator}
%For Lispish to be truly useful, its translator would need to have JavaScript validation in place. The parser used for validation could flag, for example, at which point there is an error in the generated JavaScript code and therefore simplify the task of debugging the translated code. 
%
%JavaScript that is not syntactically correct due to the faulty input Lispish is only going to decrease the productivity of the developer, which goes against the core idea of building an abstraction over an imperative language.
