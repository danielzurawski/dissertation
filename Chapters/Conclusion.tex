\chapter{Conclusions and Future Work}

The project's conclusions should list the key things that have been learnt as a consequence of engaging in your project work. For example, ``The use of overloading in C++ provides a very elegant mechanism for transparent parallelisation of sequential programs'', or ``The overheads of linear-time n-body algorithms makes them computationally less efficient than $O(n log n)$ algorithms for systems with less than 100000 particles''. Avoid tedious personal reflections like ``I learned a lot about C++ programming...'', or ``Simulating colliding galaxies can be real fun...''. It is common to finish the report by listing ways in which the project can be taken further. This might, for example, be a plan for turning a piece of software or hardware into a marketable product, or a set of ideas for possibly turning your project into an MPhil or PhD.

\section{Future work}
\subsection{Parser}
In order for Lispish translator to be a true compiler, it would need a parser that can decide whether the input string, that is the Lispish program, is in fact a correct one. As mentioend in the section above, it does not provide any error detection facility and this therefore would be the first step for proper error handling. 
\subsection{JavaScript validator}
For Lispish to be trully useful, its translator would have to have a JavaScript validator in place. 
JavaScript that is not syntactically correct due to the faulty input Lispish is only going to decrease the productivity of the developer, which goes against the core idea of building an abstraction over an imperative language.