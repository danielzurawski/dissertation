\chapter{Background}
This section provides throughout background research of the domain of functional programming, Lisp and JavaScript that lead me to the rational behind Lispish design decisions. 

It also covers existing implementations of Lisp to JavaScript translators and covers the professional issues that need to be taken into consideration in this project. 

\section{Bridging the gap between functional and imperative paradigm}

\subsection{A Comparison Between the Functional and Imperative Programming Paradigms}
Functional programming is a programming paradigm that differs from imperative programming in a way that it focuses solely on evaluating functions, where one input always results in the same output for a given input (referential transparency). In imperative programming, this notion is not always true, as imperative programming focuses a lot more on modifying the state of the application as it runs. To make referential transparency, pure functional languages try to avoid using state and mutability, by ensuring that side effects that could introduce state changes are not possible.

An example of state is preserving results in variables for later access by other parts of the program. A side effect may result from many different operations such as variable assignments, input or output operations and anything that allows two parts of the program to access the same resource at the same time.

Due to the increase of the demand for parallelisation, as more processing cores are added to modern CPUs, it is therefore essential that the software we write can be parallelised easily and without the risk of errors that could be caused by race conditions or deadlocks - which are all caused by the notion of mutability that is present in imperative languages.

The notion of pure functions may sound very impractical for a general purpose programming language, therefore functional languages used by practitioners such as Clojure allow state, but lexically scoped to its own function.
When state is absolutely necessary in order to improve the performance of an application or expose variable to other parts of the program, Clojure allows for so called "atoms", that improve on the classical notion of a variable, as it is still immutable, but instead an atomic swap operation of the content is performed whenever want to override the original state.

The property of immutability is also preserved for data structures, as each time a data structure is modified, a new copy of such structure is retained therefore leaving the old one in tact. This allows for much better parallelisation, as one part of the program may never modify the same data as the other part of the program, which would lead to inconsistent state.

\section{Programming languages involved}
In order to complete this project, it is necessary not only to understand the two different programming paradigms, but also the specific features of each of the languages involved - Clojure and JavaScript.
We will use Clojure as an example of a functional language, as Lispish is a subset of Clojure and the compiler itself is also programmed in Clojure.

\subsection{Clojure}
Clojure~\cite{Clojure:2013:Site} is a functional language, which is implemented as a dialect of Lisp and primarily targets the Java Virtual Machine. It can also target Microsoft's Common Language Runtime, which is the virtual machine for the .NET Framework through Clojure's sub-project \textbf{clojure-clr}~\cite{clojure-clr}. 
It also targets JavaScript by means of \texttt{ClojureScript}~\cite{ClojureScript:2013:Site}, which is a subset of Clojure that compiles to JavaScript. 

Clojure is a powerful abstraction over standard Java, which as of today does not provide lambdas and any of the functional constructs that Clojure does, including immutability and treating code as data.

\subsubsection{Lisp}

Lisp is amongst one of the worlds oldest family of programming languages, that has developed several dialects since the original Lisp was published in 1958-1960 by John McCarthy \cite{lisp-mccarthy}.
Lisp languages differ from other programming languages by a few original concepts, notably treating code as data, s-expressions, parenthesized Polish prefix notation and lambda expressions.

The exact expansion of the Lisp acronym is List Processing, which has its practical reasons -- Lisp source code is written as lists, formally known as S-expressions~\cite{s-expression}\cite{s-expression-wiki}.

To illustrate how a valid s-expression would look like compared to an equivalent Java expression, figure~\ref{fig:norvig-lisp-comparison} shows an excerpt from Peter Norvig's \textit{A Retrospective on Paradigms of AI Programming}~\cite{retrospective.norvig}. The figure illustrates the comparison of complexity in defining a lambda function in Lisp and in Java. 
As we can see, Lisp is not only a lot clearer syntactically, but it is also shorter and accomplishes the same goal.  

\begin{figure}[ht]

\begin{quote}
"[...] Java has anonymous classes, which serve some of the purposes of closures, although in a less versatile way with a more clumsy syntax. In Lisp, we can say \texttt{(lambda (x) (f (g x)))} where in Java we would have to say

\begin{verbatim}

new UnaryFunction() { 
  public Object execute(Object x) {
    return (Cast x).g().f();
  }
}
\end{verbatim}
where Cast is the real type of x. This would only work with classes that observe the UnaryFunction interface (which comes from JGL and is not built-in to Java). [...]"
\end{quote}

\caption{Excerpt from Peter Norvig's \textit{Paradigms of Artificial Intelligence Programming: Case Studies in Common Lisp}. Comparison of a Lisp and Java lambda function declaration.\cite{PAIP.Norvig}}
\label{fig:norvig-lisp-comparison}
\end{figure}


The article also shows an interesting view on the recent advancements of other languages compared to the ancient Lisp and it's a retrospective to the \textit{Paradigms of Artificial Intelligence Programming: Case Studies in Common Lisp}~\cite{PAIP.Norvig}. This is Peter Norvig's book about Aritficial Intelligence where all of the examples have been programmed in Common Lisp.

The simplicity and conciseness of Lisp syntax has been the main motivating factor to choose Clojure (being a functional language of the Lisp dialect) as the basis for the Lispish language. 

\subsubsection{Portability}
Due to the fact that Clojure targets the Java Virtual Machine, programs written in this language can be executed in any environment where the Java Runtime Environment is installed by means of executing Clojure programs packaged as JAR files, given that they have been packaged to include Clojure itself.  

Clojure programs can co-operate with Java applications due to its great interoperability. They can be imported into Java programs as aforementioned JAR files.

Clojure can also access all of the core Java static classes/methods, making it a very powerful abstraction over Java. It is not only because it's a very portable functional language that works with immutable data structures, but also because it gives access to the wealth of Java libraries and the entire JVM eco-system.

\subsection{JavaScript}
JavaScript is an interpreted, dynamically typed, prototyped, functional programming language that originated from the ECMAScript language in 1995. It was originally intended as a client side scripting language for web browsers, but it has since evolved to an extent where well-known corporations such as Microsoft support it as a language of choice for deploying web applications on their cloud-service offerings \cite{nodejs.WindowsAzure}. This is due to its rich support for multiple programming styles, including functional programming. More on this below in section ~\ref{portability} (JavaScript: Portability).  

\subsubsection{JavaScript Performance}
The creation of the V8 Google JavaScript Engine made JavaScript stand out from other dynamic languages by making it significantly faster than for e.g. Python ~\cite{JavaScriptVSPython3} or Ruby ~\cite{JavaScriptVSRuby}.

Due to the fact that Lispish compiles to JavaScript, the generated code can be treated with various optimisation techniques (such as the Google Closure compiler that minimises and optimises the code) by compiling the human-readable to unreadable but highly optimised JS code.

\subsubsection{Portability}\label{portability}
JavaScript interpreters are present on majority of consumer devices and are present in all of the modern web browsers. It is the basis of Rich Internet Applications and is now not only present on the front-end of the web browser, but also serves as a language of choice on the server-side.

Most notable examples include Microsoft's adoption of \texttt{node.js} for their cloud platform Windows Azure ~\cite{nodejs.WindowsAzure}, as the basis for producing highly concurrent web applications. It enables developers to write server-side applications that operate using JavaScript both on the front-end and back-end using a single programming language.

\subsection{Compiling Lispish Using a Dialect of Lisp}
The decision to use Clojure to write a compiler for my Lisp language comes from the fact that there are large advantages to using Lisp to compile Lisp.
% TODO reference for above
The nature of Lisp and its s-expressions allow us to build efficient recursive descent parsers that can take advantage of already present functions in our implementation language, Clojure.

There is a number of complexities that we would encounter when trying to implement a Lisp compiler using a non-Lisp imperative language such as C.
These include having to determine if a given expression is an s-expression (list) or a symbol, breaking the input down into its atomic form of tokens, building a Parse Tree (ST) or an Annotated Syntax Tree (AST).
In our case, our input s-expressions in their prefix notations can be treated as a parse tree and thanks to the in-built functions we can greatly simplify the compiler.

For example, any s-expression can be essentially type-checked using the in-built \texttt{(symbol? )} or \texttt{(list? )} to determine if the given s-expression yields a symbol or a list of expressions. If an input is a list, that means we have encountered another s-expression and each element in the list has to be separately evaluated.

Modern dialects of Lisp, such as Clojure, target the Java Virtual Machine making them very portable and pluggable into an existing Java applications.
Other Lisp languages are very often compiled to another target language, such as C or JavaScript that can be then run on a variety of machines. Notable examples of Lisp to C compilers include Stalin \cite{stalin} (STAtic Language ImplementatioN) and Chicken \cite{chicken}. 

\section{JavaScript as the Target Language for the Lispish Language}
The rationale behind selecting JavaScript as the target language is the fact that JavaScript can be executed on almost all Internet-enabled devices.

Our small dialect of Lisp (Lispish) language will allow generating pluggable JavaScript code.
From this follows that applications written in Lispish can be executed in environments where the JVM or Clojure is not present as the generated code will be standard JavaScript.
In theory, our language could even be used as a Domain Specific Language (DSL) for JavaScript applications, as long as the code would be evaluated by our compiler in a Clojure JVM environment.

JavaScript offers a great opportunity as a target language for any high-level programming language primarily due to two reasons - its portability and performance.

\subsection{Similarities}
Similarly to other Lisp languages, JavaScript supports the functional paradigm and provides full support for lambda expressions. It also provides a great deal of flexibility with its support for prototype-based object-orientation which could be compared to Lisp macros used for extending the language.

\begin{figure}[ht]
\begin{verbatim}
// Attach event listener to the argument
var assign-event-listener = function(x) {
  x.addEventListener("load", function() {
    alert("All done");
  }, false)
};
\end{verbatim}
\caption{JavaScript event listener function}
\label{fig:js-attach-event-listener}
\end{figure}

Figure~\ref{fig:js-attach-event-listener} illustrates a stored function that takes a reference to a web browser's window as an argument and attaches an event listener to it. The listener then takes two arguments, a string describing the event (here "load") and the callback function (here an anonymous function that displays an alert "All done") that gets displayed when the desired event is triggered.

\begin{figure}[ht]
\begin{verbatim}
(defn assign-event-listener [window]
	"Attach event listener to the argument"
	(window (addEventListener
            	"load"
                (fn [x] (alert "All done")))))
\end{verbatim}	
\caption{Possible Lisp equivalent of event listener function}
\label{fig:lispish-attach-event-listener}
\end{figure}

Similarly, the equivalent expressions written in Lisp could look like on figure ~\ref{fig:lispish-attach-event-listener}, which is arguably a lot more readable. Nonetheless, both expressions look alike. 

Both expressions make use of nested functions and thus, take advantage of lambda calculus. 

As lambda functions are inherent to programming applications in a functional language, the design and implementation of the translator will make an extensive use of lambda functions.

\section{Existing Lisp to JS Compilers}\label{existing-implementations}
There exists a number of similar projects, each of which tries to solve the problem in a slightly different way. Nevertheless, there exists only one mature compiler that provides a complete support of all of its source language constructs, which is ClojureScript.

\subsection{ClojureScript}
ClojureScript~\cite{ClojureScript:2013:Site} is a Clojure to JavaScript compiler that generates code that can be executed in the browser. Although there are examples of companies using ClojureScript for their production applications, it is difficult to operate as it requires to execute a chain of operations, including starting a JavaScript program before the Clojure code can be compiled.

ClojureScript also takes the idea further and utilises the Google Closure compiler to optimise the code but this approach suffers from the possibility of breaking the JavaScript code that was compiled from Clojure. However, the ClojureScript project page claims, that the JavaScript generated is compatible with the \texttt{advanced} mode of the Google Closure compiler.

\subsection{Outlet}
Outlet~\cite{Outlet:2013:Site} is a Lisp-like programming language that compiles to JavaScript. Its compilation is interesting in that the compiler itself is written in Outlet, only after it is bootstrapped by a JavaScript interpreter.
The bootstrapping interpreter is implemented using grammar rules similar to Backus-Naur Form context-free grammars ~\cite{RecursiveDescentParserJS:2013:Site}, thus ensuring that the initial interpretation of the main constructs is correct. This approach provides a solid foundation for then implementing rest of the constructs of the Outlet language, in Outlet.

Outlet does not provide the ability to define macros, thus there is no way to dynamically extend the language without modifying the compilers source, which is a big feature of a Lisp language.

\subsection{LiScript}
LiScript~\cite{LiScript:2013:Site} is yet another Lisp language that compiles to JavaScript. It supports roughly 20 forms, 13 of which replace the regular binary and arithmetic operations of \texttt{$>$}, \texttt{$<$} etc. and the remaining 7 are forms such as \texttt{if}, anonymous function \texttt{fn} and iterating constructs such as \texttt{iter} and \texttt{while}.

LiScript is implemented in JavaScript and it is surprisingly lightweight. The entire implementation is around 100 lines of code. Nonetheless, it can generate a readable and, most importantly, executable JavaScript code. 

LiScript allows defining new language constructs by means of macros, a special form \texttt{defmacro}. This allows for building new forms from arbitrary strings, as input to the defmacro form first modifies the code and only then evaluates it. 

\subsection{clojurejs}
ClojureJS~\cite{ClojureJS:2011:Site} is a small subset of Clojure to JavaScript compiler.
ClojureJS takes on an approach different to the preceding implementations, as the compiler is written in Clojure. 
It is a hand-written recursive descent parser that requires a running Clojure environment in order to evaluate the input source code.

ClojureJS is perhaps the second best implementation of Clojure to JavaScript compilation, with its support for macros and a lot larger subset of Clojure than, for instance, LiScript. 
It is, arguably, a non extensible implementation of a translator due to its not strictly idiomatic code base. From usability perspective, it does not provide any ease-of-use features, such as being able to generate an output \texttt{.js} file out of a given source input. 

\section{Professional and ethical issues}
Throughout the report, I will make sure that I am not violating the British Computer Society Code of Conduct and Code of Good Practice and that I have applied their principles throughout the project.

All possible efforts shall be made to properly state and reference communicating someone else's ideas.
