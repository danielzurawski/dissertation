\chapter{Evaluation}
The implementation of Lispish to JavaScript was supposed to serve as an example of how in a simple way, a programming language translator can be implemented.

Due to its functional nature and immutability, it does not suffer from errors caused by inconsistent state, as there is no state. The compiler will always produce a result. The result however is not guaranteed to be correct, if an incorrect input has been provided. 

Appendix [REFERENCE HERE] provides a set of unit tests that include three real applications of our translator by translating a Clojure Fibonacci, factorial and Ackermann functions to its equivalent JavaScript functions, as well as smaller tests that check the correctness of single forms.
The unit tests included in the above mentioned appendix all successfully pass.
The compiler also successfully compiles a given input file to an output file with name of our choice. 

\section{Missing parts}
\subsection{Error handling}
The compiler does not provide any facility for error reporting during the compilation.

The compiler does not have any means of validating the JavaScript code. This could be incorporated by means of bundling a JavaScript validator that could simply analyse the code before it's served to an output file. This was however not part of the initial design and due to time constraints has not been implemented.

The biggest issue with the compiler is that it does not actually parse the input string before the compilation is performed. This caveat removes the posibility of determining if the input source, that is Lispish, is actually valid. 
Providing an invalid Lispish source code would still result in a JavaScript output, but the generated code would be malformed and would not execute in a browser. This is both true for semantical, as well as syntactical errors.